%&latex

\documentclass{beamer}
\usepackage{listings}
\usefonttheme{serif}
% Customize slide appearance
\mode<presentation>
{
  \usetheme{Warsaw}
  \setbeamercovered{transparent}
}


\usepackage[english]{babel}
\usepackage{times}

\begin{document}

\title{Algebra and Join Minimization}
\author{John Clara, Danyang Zhang}
\date[WI 2016]{Winter 2016.}

\subject{Algebra and Join Minimization}

\begin{frame}
  \titlepage
\end{frame}

\begin{frame}
  \frametitle{How to Optimize Queries}
  \begin{itemize}
    \item Perform different mappings to reduce rows
    \item Answer variables cannot change (least degree of freedom)
    \item Constants cannot change (least degree of freedom)
    \item Everything else is fair game!
  \end{itemize}
\end{frame}

\section{Join Minimization}
\begin{frame}
  \frametitle{Example 1}
  What are all the books by the person who wrote ``Twilight"?
\end{frame}

\begin{frame}[fragile]
  \frametitle{Example 1}
  What are all the books by the person who wrote ``Twilight"?\\
  \hfill \\
\begin{verbatim} 
  SELECT b1.title
  FROM Book b1, Book b2, Book b3
  WHERE b1.author = b2.author AND
        b3.author = b2.author AND
        b3.title = "Twilight";
\end{verbatim}

\end{frame}

\begin{frame}[fragile]
  \frametitle{Example 1}
  What are all the books by the person who wrote "Twilight"?\\
  \hfill \\
\begin{verbatim} 
  SELECT b1.title
  FROM Book b1, Book b2, Book b3
  WHERE b1.author = b2.author AND
        b3.author = b2.author AND
        b3.title = "Twilight";
\end{verbatim}
  \hfill \\
  \begin{tabular}{ c | c c }
  Book & title & author \\
  \hline
   b1 & \textcolor{red}{d} & a \\
   b2 & -         & a \\
   b3 & ``Twilight" & a \\
  \end{tabular}
  \begin{tabular}{ c | c}
  answer & title \\
  \hline
   & \textcolor{red}{d}\\
  \end{tabular}

  Can we map first row to any rows?
\end{frame}

\begin{frame}
  \frametitle{Example 1}
  What are all the books by the person who wrote "Twilight"?\\
  \hfill \\
  \begin{tabular}{ c | c c }
  Book & title & author \\
  \hline
   b1 & \textcolor{red}{d} & a \\
   b2 & -         & a \\
   b3 & ``Twilight" & a \\
  \end{tabular}
  \begin{tabular}{ c | c}
  answer & title \\
  \hline
   & \textcolor{red}{d}\\
  \end{tabular}
  \hfill \\
  Map second row to some row?
\end{frame}

\begin{frame}
  \frametitle{Example 1}
  What are all the books by the person who wrote "Twilight"?\\
  \hfill \\
  \begin{tabular}{ c | c c }
  Book & title & author \\
  \hline
   b1 & \textcolor{red}{d} & a \\
   b3 & ``Twilight" & a \\
  \end{tabular}
  \begin{tabular}{ c | c}
  answer & title \\
  \hline
   & \textcolor{red}{d}\\
  \end{tabular}
  \hfill \\
  Map second row to some row?
\end{frame}

\begin{frame}[fragile]
  \frametitle{Example 1}
  What are all the books by the person who wrote "Twilight"?\\
  \begin{tabular}{ c | c c }
  Book & title & author \\
  \hline
   & \textcolor{red}{d} & a \\
   & "Twilight" & a \\
  \end{tabular}
  \begin{tabular}{ c | c}
  answer & title \\
  \hline
   & \textcolor{red}{d}\\
  \end{tabular}
  \hfill \\
\begin{verbatim}
  SELECT b1.title
  FROM Book b1, Book b2
  WHERE b1.author = b2.author AND
        b2.title = "Twilight";
\end{verbatim}        
\end{frame}

\begin{frame}[fragile]
  \frametitle{Example 2}
\begin{verbatim}  
SELECT t1.A, t2.B, t4.C
FROM R t1, R t2, R t3, R t4, R t5
WHERE t3.A=t4.A AND
  t2.B=t3.B AND
  t1.C=t2.C AND
  t3.C=t5.C AND
  t3.A=t5.A;
\end{verbatim}  
\end{frame}

\begin{frame}[fragile]
  \frametitle{Example 2}
\begin{verbatim}  
SELECT t1.A, t2.B, t4.C
FROM R t1, R t2, R t3, R t4, R t5
WHERE t3.A=t4.A AND
  t2.B=t3.B AND
  t1.C=t2.C AND
  t3.C=t5.C AND
  t3.A=t5.A;
\end{verbatim} 
  \begin{tabular}{ c | c c c}
  R & A & B & C \\
  \hline
  t1 & \textcolor{red}{a}  & -  & c1 \\
  t2 & -  & \textcolor{red}{b}  & c1 \\
  t3 & a1 & \textcolor{red}{b} & c2 \\
  t4 & a1 & - & \textcolor{red}{c} \\
  t5 & a1 & - & c2 \\
  \end{tabular}
  \begin{tabular}{ c | c c c}
  answer & A & B & C \\
  \hline
   & \textcolor{red}{a}& \textcolor{red}{b}& \textcolor{red}{c}\\
  \end{tabular}
\end{frame}

\begin{frame}
  \frametitle{Example 2}
  \begin{tabular}{ c | c c c}
  R & A & B & C \\
  \hline
  t1 & \textcolor{red}{a}  & -  & c1 \\
  t2 & -  & \textcolor{red}{b}  & c1 \\
  t3 & a1 & \textcolor{red}{b} & c2 \\
  t4 & a1 & - & \textcolor{red}{c} \\
  t5 & a1 & - & c2 \\
  \end{tabular}
  \begin{tabular}{ c | c c c}
  answer & A & B & C \\
  \hline
   & \textcolor{red}{a}& \textcolor{red}{b}& \textcolor{red}{c}\\
  \end{tabular}
  \hfill \\
  Can we reduce any rows?
\end{frame}

\begin{frame}
  \frametitle{Example 2}
  Reduce t5
  
  \begin{tabular}{ c | c c c}
  R & A & B & C \\
  \hline
  t1 & \textcolor{red}{a}  & -  & c1 \\
  t2 & -  & \textcolor{red}{b}  & c1 \\
  t3 & a1 & \textcolor{red}{b} & c2 \\
  t4 & a1 & - & \textcolor{red}{c} \\
  \end{tabular}
  \begin{tabular}{ c | c c c}
  answer & A & B & C \\
  \hline
   & \textcolor{red}{a}& \textcolor{red}{b}& \textcolor{red}{c}\\
  \end{tabular}
\end{frame}

\begin{frame}
  \frametitle{Example 2}
  
  TODO reduce t2 to t3
  
  \begin{tabular}{ c | c c c}
  R & A & B & C \\
  \hline
  t1 & \textcolor{red}{a}  & -  & c1 \\
  t2 & -  & \textcolor{red}{b}  & c1 \\
  t3 & a1 & \textcolor{red}{b} & - \\
  t4 & a1 & - & \textcolor{red}{c} \\
  \end{tabular}
  \begin{tabular}{ c | c c c}
  answer & A & B & C \\
  \hline
   & \textcolor{red}{a}& \textcolor{red}{b}& \textcolor{red}{c}\\
  \end{tabular}
\end{frame}

\begin{frame}
  \frametitle{How to Chase}
  \begin{figure}[!htp]
\centering
\subfloat{\includegraphics[scale=.80]{chase}}
\end{figure}
\end{frame}

\begin{frame}
Degree of freedom (dof): $wildcard > nonanswer~var > answer~var = const$

Trick: replace the one with higher dof with the lower dof.
\end{frame}

\begin{frame}
  \frametitle{Example 2}
  Dependencies: $F = \{AC \rightarrow B, B \rightarrow C, C \rightarrow A \}$\\
  \begin{tabular}{ c | c c c}
  R & A & B & C \\
  \hline
  & \textcolor{red}{a}  & -  & c1 \\
  & -  & \textcolor{red}{b}  & c1 \\
  & a1 & \textcolor{red}{b} & - \\
  & a1 & - & \textcolor{red}{c} \\
  \end{tabular}
  \begin{tabular}{ c | c c c}
  answer & A & B & C \\
  \hline
   & \textcolor{red}{a}& \textcolor{red}{b}& \textcolor{red}{c}\\
  \end{tabular}
\end{frame}

\begin{frame}
  \frametitle{Example 2}
  Dependencies: $F = \{AC \rightarrow B, B \rightarrow C, C \rightarrow A \}$\\
  Use $B \rightarrow C$\\
  \begin{tabular}{ c | c c c}
  R & A & B & C \\
  \hline
  & \textcolor{red}{a}  & -  & c1 \\
  & -  & \textcolor{red}{b}  & c1 \\
  & a1 & \textcolor{red}{b} & - \\
  & a1 & - & \textcolor{red}{c} \\
  \end{tabular}
  \begin{tabular}{ c | c c c}
  answer & A & B & C \\
  \hline
   & \textcolor{red}{a}& \textcolor{red}{b}& \textcolor{red}{c}\\
  \end{tabular}
\end{frame}

\begin{frame}
  \frametitle{Example 2}
  Dependencies: $F = \{AC \rightarrow B, B \rightarrow C, C \rightarrow A \}$\\
  Use $C \rightarrow A$\\
  \begin{tabular}{ c | c c c}
  R & A & B & C \\
  \hline
  & \textcolor{red}{a}  & -  & c1 \\
  & -  & \textcolor{red}{b}  & c1 \\
  & a1 & \textcolor{red}{b} & c1 \\
  & a1 & - & \textcolor{red}{c} \\
  \end{tabular}
  \begin{tabular}{ c | c c c}
  answer & A & B & C \\
  \hline
   & \textcolor{red}{a}& \textcolor{red}{b}& \textcolor{red}{c}\\
  \end{tabular}
\end{frame}

\begin{frame}
  \frametitle{Example 2}
  Dependencies: $F = \{AC \rightarrow B, B \rightarrow C, C \rightarrow A \}$\\
  Eliminate rows\\
  \begin{tabular}{ c | c c c}
  R & A & B & C \\
  \hline
  & \textcolor{red}{a}  & -  & c1 \\
  & \textcolor{red}{a}  & \textcolor{red}{b}  & c1 \\
  & \textcolor{red}{a}  & \textcolor{red}{b} & c1 \\
  & \textcolor{red}{a}  & - & \textcolor{red}{c} \\
  \end{tabular}
  \begin{tabular}{ c | c c c}
  answer & A & B & C \\
  \hline
   & \textcolor{red}{a}& \textcolor{red}{b}& \textcolor{red}{c}\\
  \end{tabular}
\end{frame}

\begin{frame}
  \frametitle{Example 2}
  Dependencies: $F = \{AC \rightarrow B, B \rightarrow C, C \rightarrow A \}$\\
  Can we use any Dependencies?\\
  \begin{tabular}{ c | c c c}
  R & A & B & C \\
  \hline
  & \textcolor{red}{a}  & \textcolor{red}{b} & c1 \\
  & \textcolor{red}{a}  & - & \textcolor{red}{c} \\
  \end{tabular}
  \begin{tabular}{ c | c c c}
  answer & A & B & C \\
  \hline
   & \textcolor{red}{a}& \textcolor{red}{b}& \textcolor{red}{c}\\
  \end{tabular}
\end{frame}
\begin{frame}[fragile]
  \frametitle{Example 2}
  Dependencies: $F = \{AC \rightarrow B, B \rightarrow C, C \rightarrow A \}$\\
  \begin{tabular}{ c | c c c}
  R & A & B & C \\
  \hline
  & \textcolor{red}{a}  & \textcolor{red}{b} & - \\
  & \textcolor{red}{a}  & - & \textcolor{red}{c} \\
  \end{tabular}
  \begin{tabular}{ c | c c c}
  answer & A & B & C \\
  \hline
   & \textcolor{red}{a}& \textcolor{red}{b}& \textcolor{red}{c}\\
  \end{tabular}
  \hfill \\
\begin{verbatim} 
  SELECT r1.A, r1.B, r2.C
  FROM R r1, R r2
  WHERE r1.a = r2.a;
\end{verbatim}  
\end{frame}

\section{Reference}
\begin{frame}[fragile]
\frametitle{Reference}
\begin{enumerate}
\item ``\textit{Database Systems Concepts}'' by Silberschatz, Korth and Sudarshan, 6th edition, McGraw-Hill.
\end{enumerate}
\end{frame}

\end{document}
