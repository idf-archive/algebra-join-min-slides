%&latex

\documentclass{beamer}
\usepackage{listings}
\usefonttheme{serif}
% Customize slide appearance
\mode<presentation>
{
  \usetheme{Warsaw}
  \setbeamercovered{transparent}
}


\usepackage[english]{babel}
\usepackage{times}

% You can add any graphics to every slide by following command:
% \logo{\includegraphics{logo.eps}}

% Uncomment this, if you want the table of contents before each subsection.
% However, to edit slides in TeXWord avoiding this feature is good idea.
% \AtBeginSubsection[]
% {
%   \begin{frame}<beamer>
%     \frametitle{Outline}
%     \tableofcontents[currentsection,currentsubsection]
%   \end{frame}
% }

% If you wish to uncover everything in a step-wise fashion, uncomment
% the following command:
%\beamerdefaultoverlayspecification{<+->}

\begin{document}

% Title Data. We keep it after \begin{document}
% to enable editing text in BaKoMa TeX Word.

\title{Algebra and Join Minimization}
\author{John Clara, Danyang Zhang}
\date[WI 2016]{Winter 2016.}

\subject{Algebra and Join Minimization} % Should be passed to PDF [YNI]

\begin{frame}
  \titlepage
\end{frame}

\section{Relational Calculus Overview}
\begin{frame}
  \frametitle{Atoms}
  \begin{itemize}
  \item $t \in R$, where $t$ is a tuple variable, $R$ is a relation
  \item $t(x) \Theta u(y)$, where $t, u$ are tuple variables, $x, y$ are the attributes. $\Theta$ is comparison operator ($<, \leq, =, \neq, \geq, >$).
  \item $t(x) \Theta c$, where $c$ is a constant.
  \end{itemize}
\end{frame}

\begin{frame}
  \frametitle{Formulae}
  \begin{itemize}
  \item An atom $P$ is a formula
  \item $\neg P, P_1 \wedge P_2, P_1 \vee P_2$
  \item $P_1 \rightarrow P_2$
  \item $P(t)$ contains a free tuple $t$, and.
  \begin{align*}
  &\exists t \in R [P(t)]\\
  &\forall t \in R [P(t)]
  \end{align*}
  \end{itemize}
\end{frame}

\section{Reference}
\begin{frame}[fragile]
\frametitle{Reference}
\begin{enumerate}
\item ``\textit{Database Systems Concepts}'' by Silberschatz, Korth and Sudarshan, 6th edition, McGraw-Hill.
\end{enumerate}
\end{frame}

\end{document}
